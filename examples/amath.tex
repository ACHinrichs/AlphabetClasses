\documentclass{ctext}
\selectlanguage{english}

\usepackage{amath}
\title{\texttt{amath} Example}
\subtitle{Using \texttt{ctext}}
\author{Alexander Bartolomey}
\date{06.06.2017}

\begin{document}
\maketitle
\tableofcontents*
\section{Introduction}
\textsc{aMath} features lots of neat short commands for symbols and mathematical structures. As seen below, most of them are available through \textsc{amsmath}.
\section{Equation Styling}
\textsc{aMath} imports \textsc{amsmath} with \verb|[fleqn]| option to flush all equations in \emph{align} environments (or other environments serving the same purpose) to the left.
\section{Symbols \& Abbreviations}
For mathematical symbols (those I use most of the time), I added macros like explained in Table \ref{tab:symb}:
\begin{table}[H]
  \centering
  \begin{tabular}{c|c|c}
    Inline Math & Command & Comment \\
    \hline
    \(\N\) & \verb|\N| & To display \(\N_0\), use \verb|\N_0|\\
    \(\Z\) & \verb|\Z| & \\
    \(\R\) & \verb|\R| & \\
    \(\Q\) & \verb|\Q| & \\
    \(\C\) & \verb|\C| & \\
    \(\F\) & \verb|\F| & For Prime Fields.\footnotemark[1]\\
    \hline
    \(\GL\) & \verb|\GL| & General Linear Group \\
    \(\id\) & \verb|\id| & Identity \\
    \(\diff{x}\) & \verb|\diff{x}| & Analysis: Derivate function\\
    \hline
    \(\divides\) & \verb|\divides| & Operator \\
    \(\property\) & \verb|\property| & Separator for Sets\footnotemark[2] \\
    \hline
    \(\Var\) & \verb|\Var| & Combinatorics: Variations \\
    \(\Perm\) & \verb|\Perm| & Combinatorics: Permutation \\
    \(\MComb\) & \verb|\v| & Combinatorics: Multicombination \\
    \(\Comb\) & \verb|\Comb| & Combinatorics: Combinations \\
    \hline
    \(\dim\) & \verb|\dim| & Dimension function \\
    \(\Im\) &  \verb|\Im| & Image of certain function \\
    \hline
    \(\modulo{X}{m}\) & \verb|\modulo{<>}{<>}| & Modulus (Sets) operator \\
    \hline
    \(\Pot(X)\) & \verb|\Pot X| & Power set \\
    \(\Map(X,Y)\) & \verb|\Map(X,Y)| & Set of Maps from \(X\) to \(Y\) \\
    \(\Bin\) & \verb|\Bin| & \\
    \(\falls\) & \verb|\falls| & To be used in case conditions \\
    \(\charakteristik\) & \verb|\charakteristik| & Characterisic of a field
  \end{tabular}
  \caption{Symbols \& Abbreviation Commands from \textsc{aMath}}
  \label{tab:symb}
\end{table}
\footnotetext[1]{GitHub User \href{https://github.com/ACHinrichs}{\texttt{@ACHinrichs}} found this command to be malicious with other classes. When he merged this class into his HomeworkAssignment class, he renamed it to \texttt{\textbackslash Primes}}
\footnotetext[2]{\texttt{\textbackslash property} and \texttt{\textbackslash divides} share the same definition but are intended to be used differently}
\section{Operators}
\subsection{Quantifiers}
\textsc{aMath} restyles both universal and existential quantifiers to use up more space. It also adds a large version for display mode math, available through \verb|\bigforall| and \verb|\bigexists|. In general, they look like this:
\[
  \bigforall x \in \R \bigexists y \in \R : x^2 = y
\]
The same formula in inline math mode: \(\bigforall x \in \R \bigexists y \in \R : x^2 = y\)

The adjustment to the default quantifiers looks like this: \[
  \forall x \in \R \exists y \in \R : x^2 = y
\]
\subsection{Restriction}
A macro to display a restriction to function or a map is provided by using
\begin{verbatim}
  \restr{<function>}{<new source>}{<new target>}.
\end{verbatim}
In math, it looks like this:
\[
  \restr{f}{U}{V}: U \rightarrow V, x \mapsto f(x).
\]
\section{Functions}
\subsection{Absolute value}
Using \verb|\abs{<>}|, you can display an absolute value.
\[
  \abs{x-y} = \text{"Distance of x and y"}.
\]
\subsection{Rounding functions}
For rounding up and down, \textsc{aMath} provides \verb|\ceil{<>}| and \verb|\floor{<>}|.
\[
  \floor{3.5} = 3, \quad \ceil{3.5} = 4.
\]
\section{Utilities}
\subsection{Relations}
You can define your own relation by using \verb|\rel{<>}|. For example, reflexivity of a relation \(m\) looks like this:
\[
  x \rel{m} y \Rightarrow x = y.
\]
\subsection{Large brackets}
Wrapping brackets around fractions can be a pain, therefore \textsc{aMath} provides the \verb|\labra{<>}| command, which sets fitting brackets around the given argument.
\[
  \labra{\frac{n+2}{n+1}}
\]
\subsection{Vectors}
You can create a vector without using any environments yourself by using \verb|\colvec{<>}| and seperating the rows with a general linebreaking \LaTeX\ operator.
\[
  a := \colvec{a_1\\a_2\\a_3}
\]
\end{document}
