\documentclass{ctext}
\usepackage{atask}
\usepackage{lipsum}

\title{\texttt{atask} Example}
\subtitle{Using \texttt{ctext}}
\author{Alexander Bartolomey}
\team{AlphabetClasses}
\date{06.06.2017}
\begin{document}
\selectlanguage{english}
\maketitle
\tableofcontents*

\section{Introduction}
\atask introduces two (with options four) new sectioning commands, \verb|\aufgabe{<>}| and \\ \verb|\teilaufgabe{<>}|. Using the starred version of the commands lets you manually increment the counters to a given state. Keep in mind that counters start from 0, so setting the counter with a starred version requires you to subtract 1 from the desired counter value as it is displayed.
\vfill
\section{Commands}
\begin{verbatim}
  \aufgabenprefix{} | \teilaufgabenprefix{}
\end{verbatim}
Defines the actual name of the created section abbreviation.
Defaults to a translation of "`Task"' or "`Subtask"' in German or English, but can be used as a completely different type of section command.
\begin{verbatim}
  \aufgabensuffix{} | \teilaufgabensuffix{}
\end{verbatim}
Behaves like \verb|\aufgabenprefix{}| (\verb|\teilaufgabenprefix{}|) except defaulting to an emtpy string and \textbf{appending} its value to the section title
\begin{verbatim}
  \marginBeforeAufgabe{} | \marginBeforeTeilaufgabe{}
  \marginAfterAufgabe{}  | \marginAfterTeilaufgabe{}
\end{verbatim}
Determines the spacing before and after tasks and subtasks.
\begin{verbatim}
  \aufgabencounter{} | \teilaufgabencounter{}
\end{verbatim}
Whenever you want to replace the default counter for tasks and subtasks, use these commands. These will replace the counters with static values, otherwise use a counter included by the specific command (\verb|\thesection|,\verb|\theaufgabe| for \verb|\aufgabe|, \verb|\thesubsection|, \verb|\theteilaufgabe| for subtasks) or the default values (\verb|\atask@aufgabe|, \verb|\atask@teilaufgabe|) for dynamic counting.
\section{Examples}
\subsection{Example I}
For only displaying the small letter for each subtask, use
\begin{verbatim}
  \usepackage[teilaufgabeCounter=alph]{atask}
  ...
  \teilaufgabenprefix{}
  \teilaufgabensuffix{}
  \teilaufgabencounter{\alph{teilaufgabe})}
  \teilaufgabe{}
  ...
\end{verbatim}
This will be compiled as seen below.
\teilaufgabenprefix{}
\teilaufgabensuffix{}
\teilaufgabencounter{\alph{teilaufgabe})}
\aufgabe{}
\teilaufgabe{}
\lipsum[1]


\teilaufgabenprefix{\GetTranslation{teilaufgabe}}
\teilaufgabencounter{\theaufgabe.\alph{\teilaufgabe}}

\subsection{Example II}
Another wild example may be given by
\begin{verbatim}
  \aufgabenprefix{Aufgabe}
  \teilaufgabenprefix{Quatsch}
  \teilaufgabencounter{\(e^{-i\pi}\)}
  \marginBeforeAufgabe{2cm}
  \marginAfterAufgabe{1cm}
  \marginBeforeTeilaufgabe{1cm}
  \marginAfterTeilaufgabe{0.5cm}
  \aufgabe{}
  ...
  \teilaufgabe*[2]{Task}
  ...
  \teilaufgabe{}
  ...
  \aufgabe{}
  \teilaufgabe{}
  ...
\end{verbatim}
This looks like this:
\aufgabenprefix{Aufgabe}
\teilaufgabenprefix{Quatsch}
\teilaufgabencounter{\(e^{-i\pi}\)}
\marginBeforeAufgabe{2cm}
\marginAfterAufgabe{1cm}
\marginBeforeTeilaufgabe{1cm}
\marginAfterTeilaufgabe{0.5cm}
\aufgabe{}
\lipsum[1]
\teilaufgabe*[2]{Task}
\lipsum[1]
\teilaufgabe{}
\lipsum[1]
\aufgabe{}
\teilaufgabe{}
\lipsum[1-6]
\end{document}
