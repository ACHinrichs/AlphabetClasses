\documentclass{ctext}
\usepackage{amath}
\usepackage{bmath}

\title{\texttt{bmath} Example}
\subtitle{Using \texttt{ctext} \& \texttt{amath}}
\author{Alexander Bartolomey}
\date{06.06.2017}

\begin{document}
\selectlanguage{english}
\maketitle
\tableofcontents*
\section{Introduction}
\bmath defines some new environments for usage in scripts or subtasks. Environments may be chosen from the following list. Starred versions let you specify a name for the given type of environment.
\begin{itemize}
  \item \verb|definition| (available in starred version) lets you define formulas
  \item \verb|beispiel| (available in starred version) lets you state an example
  \item \verb|bemerkung| adds a remark
  \item \verb|konvention| defines a convention
  \item \verb|proposition| (available in starred version) proposes content
  \item \verb|notation| adds a notational convention
  \item \verb|satz| (available in starred version) defines a theorem
  \item \verb|korollar| defines a corollary
  \item \verb|lemma| (available in starred version) defines a lemma
  \item \verb|algorithmus| (available in starred version) marks an algorithm
  \item \verb|induktion| starts an inductional proof and puts appends a proof end mark at the end.\footnote{\texttt{\textbackslash QED} is a command defined by \bmath putting a right-flushed square in the line}
\end{itemize}
\bmath requires the \texttt{amsthm} package, which by default adds some environments, like \verb|\begin{proof}\end{proof}| which MIGHT be useful in combination with \bmath.
A minimal working example SHALL be given by:
\section{Examples}
\setcounter{aufgabe}{2}
\begin{notation}
  \bmath SHALL be displayed in small capitals.
\end{notation}
\selectlanguage{german}
\begin{satz*}{Homomorphiesatz für Mengen}
      Es sei eine Abbildung $f: X \rightarrow Y $ gegeben. Dann haben wir eine induzierte Abbildung $\bar{f}:\modulo{X}{=_f} \rightarrow Y, [x] \mapsto f(x)$, welche $f = \bar{f} \circ \text{quo}$ erfüllt. Es ist $\bar{f}$ injektiv und $\Im \bar{f} = \Im f$. Insbesondere ist
  \[\restr{\bar{f}}{\Im f}{} : \modulo{X}{=_f} \rightarrow \Im f \]
  eine Bijektion.
\end{satz*}
\selectlanguage{english}
\begin{korollar}
  \bmath is quite nice.
  \begin{proof}
    This is left for exercise purpose to the reader.
  \end{proof}
\end{korollar}
\selectlanguage{german}
\begin{notation}
  \begin{enumerate}[label=(\alph*)]
    \item Es sei ein Monoid $M$ gegeben. Für jedes $n \in \N_0$ und alle $x \in M^n$ mit $x_ix_j = x_jx_i$ für $i, j \in [1, n]$ notieren wir rekursiv
    \[\prod\limits_{i\in [1,n]} x_i := \begin{cases}
    1, & \text{falls } n = 0, \\
    (\prod_{i \in [1,n-1]} x_i)x_n, & \text{falls } n > 0
    \end{cases}\]
    \item Es sei ein abelsches Monoid $A$ gegeben. Für jedes $n \in \N_0$ und alle $x \in A^n$ notieren wir rekursiv
    \[\sum\limits_{i\in[1,n]}x_i := \begin{cases}
      0, & \text{falls } n = 0,\\
      \sum_{i \in [1, n-1]} x_i + x_n, & \text{falls } n > 0
    \end{cases}\]
  \end{enumerate}
\end{notation}
\subsection{Special case: \texttt{induktion} environment}
\selectlanguage{english}
\begin{induktion}
  \anfang \(n = 1:\) \dots
  \voraussetzung \dots
  \schritt \(n \mapsto n+1:\) \dots
\end{induktion}
\section{Markers}
\bmath adds some (pseudo-)subsectioning commands, which are not displayed in any glossary (Basically, they are just fancy-styled text). As seen in table \ref{tab:markers}, they look like this:
\begin{table}[H]
  \centering
\begin{tabular}{l|l|l}
Marker & Command & Comment \\
\hline
\anfang & \verb|\anfang| & Marks the basis of a proof by induction \\
\voraussetzung & \verb|\voraussetzung| & Marks the inductive hypothesis \\
\schritt & \verb|\schritt| & Marks the inductive step \\
\hline
\behauptung & \verb|\behauptung| & Marks a hypothesis \\
\beweis & \verb|\beweis| & Marks a proof \\
\ueberlegung & \verb|\ueberlegung| & Marks an assumption \\
\gegenbeispiel & \verb|\gegenbeispiel| & Marks a counterexample
\end{tabular}
\caption{Marker commands}
\label{tab:markers}
\end{table}
\end{document}
