\documentclass[alex,hendrik,johannes]{htext}
\usepackage{lipsum}
\title{\texttt{htext} Example}
\subtitle{Subtitle support}
\team{AlphabetClasses}
\author{Alexander Bartolomey}
\date{06.06.2017}
\begin{document}
\maketitle
This is \textsc{hText}. \hText goes bigger, better and more modern than
\textsc{cText} (on which it's originally based upon).
\section{Lorem Ipsum}
\subsection{Testing Text}
\subsubsection{Testing...}
\lipsum[1-3]
\section{Math}
\[
  \sum\limits_{k=1}^{\infty} \frac{1}{k} \rightarrow \infty
\]
This is how the harmonic series and math looks like in \hText. It's set in the
TeX clone of Adobe's Palatino. Generally, \(\lambda = X_1 \wedge X_2 \wedge X_3
\) is a boolean formula with a model. For \((1,1,1)\) the formula is true.
\section{Neat Gimmicks}
Using the class options \texttt{alex}, \texttt{hendrik} or \texttt{johannes}
will yield any of us as icons in the top right corner.

You can use the macro \texttt{\team} in the preamble to display a team in the
top left header. Note that the ``Team'' is automatically prepended. The header
however can be redefined to do anything you want using the syntax of fancyhdr,
which is required in the backend.
\end{document}
